\documentclass{article}
\usepackage[utf8]{inputenc}
\usepackage{amsmath}
\usepackage{amsfonts}
\usepackage{amssymb}
\usepackage{graphicx}
\usepackage{hyperref}
\usepackage{algorithm}
\usepackage{algpseudocode}
\usepackage{listings}
\usepackage{color}
\usepackage{float}
\usepackage{tikz}
\usepackage{tikz-qtree}
\begin{document}
\title{Computer Architecture Final 2024}
\author{Hanzuo Liu, Kunxiao Qian}
\date{\today}
\maketitle
\section*{Problem 1 True or False}
\subsection*{Part A}
3 pts each.
\subsubsection*{1}
Prefetch can not accelerate some data structures such as tree etc.
\subsubsection*{2}
Raise NotImplementedError()
\subsubsection*{3}
Double the cache capacity will always decrease the miss rate.
\subsubsection*{4}
Oblivious data algorithm could solve the problem of traditional prime-probe attack.
\subsubsection*{5}
physical register file could be a potential side channel
\subsubsection*{6}
Raise NotImplementedError()
\subsubsection*{7}
GPU will load the frequently accessed data in on-chip shared memory, at some time, this will harm the performace.
\subsection*{Part B}
4 pts.\\
list at least 4 design philosiphies of computer architecture mentioned in class.
\section*{Problem 2 In theory, theory and practice are the same}
\subsection*{Part A}
WriteBack(addr): if the addr is dirty, write back, otherwise nothing. In both case, cacheline is still valid.\\
Invalidate(addr): invalidate the cacheline. If dirty, write back.\\
\begin{lstlisting}[language=C]
//Thread 1
a = 1;
flag = 1;
\end{lstlisting}
\begin{lstlisting}[language=C]
//Thread 2
while(flag == 0){}// waiting
print a;
\end{lstlisting}
\subsubsection*{1}
if we use write-through as our replacement policy. Modify the code to make it work without introducing any unneccessary performance loss.\\
\subsubsection*{2}
change the write-through to write-back in the \textbf{1}, redo the problem.
\subsection*{Part B}
\textbf{This part is totally same as the sample exam problem 4 part A}.\\
After graduation, you start to work in a large Bird company. You are asked by your manager to 
analyze the performance of a computer machine in the lab. The machine uses a normal multicore
 processor with multi-level caches, and the main memory uses DRAM. Specifically, you 
need to measure the sequential and random data access bandwidth of the DRAM memory. 
You do not need to worry about the file system storage or virtual memory in this problem. 
\subsubsection*{1}
You manager \texttt{Big Bird} says that he expects the random bandwidth of DRAM to be much 
lower than the sequential bandwidth. Why would he expect such a result? Use the DRAM 
system structures and organizations we learned in class to provide a reason.
\subsubsection*{2}
However, after actual measurement, you find that the DRAM random access 
bandwidth is very close to the sequential bandwidth, with very small difference. The 
programs you use access the same amount of data that fit in the main memory, only with 
different access patterns (sequential vs. random). You have also confirmed that the accesses 
to the DRAM, i.e., the misses from the last-level cache, are indeed sequential and random in 
the two programs, and so it is not because of the caches. Use the DRAM system structures 
and organizations we learned in class, provide two reasons to convince your team leader that 
the result is explainable.
\section*{Problem 3 Fun with Memories}
\begin{table}[H]
\centering
\begin{tabular}{|c|c|}
\hline
Something & Some other thing  \\
\hline
Virtual Address & 48 bits  \\
Physical Address & 36 bits \\
Page Size & 4KB \\
Addressing & Byte addressing \\
\hline
\end{tabular}
\label{meomory}
\end{table}
\begin{table}[H]
\centering
\begin{tabular}{|c|c|c|c|c|c|}
\hline
Items & TLB-I & TLB-D & L1-I & L1-D & L2 \\
\hline
Capacity & 32 entries & 128 entries & 8kB & 32kB & 2MB \\
Assocaiativity & fully-associate & fully-associate & direct-mapped & 4-way & 8-way \\ 
Block Size & N/A & N/A & 64bytes & 64bytes & 64bytes \\
\hline
\end{tabular}
\label{sizes}
\end{table}
\subsection*{Part A}
Calculate the Tag, Index(for caches), Offset and table lines/sets of TLB-D, LI-D, L2. In the exam, you should fill the table.
\subsection*{Part B}
\subsubsection*{1}
In reality, what is the minimal number of SRAM bits for L1-I.
\subsubsection*{2}
Double the block size and recalculate the result in \textbf{1} ,capacity and associativity keeps same.
\subsubsection*{3}
After concluding the latency Result, \texttt{Bird} conclude them into a table shown as below.
\begin{table}[H]
\centering
\begin{tabular}{|c|c|c|c|c|}
\hline
TLB latency & TLB miss penalty & L1-D latency & L2 latency & Memory Latency \\
\hline
2  & 50  & 3 & 10 & 100 \\
\hline
\end{tabular}
\end{table}
\begin{lstlisting}[language=C]
#define D 4096
uint32_t bird[5*D];
for(int i=0;i<1000000;i++){
    for(int j=0;j<D;j+=1){
        bird[j] = bird[j+D]+bird[j+2*D]+bird[j+3*D]+bird[j+4*D];
    }
}
\end{lstlisting}
Note that since we avoid the cold miss at the begining of the program, we will run enough loops and you only need to calculate the non-compulsory misses for \texttt{Bird}.\\
Remind that you should check whether TLB and Cache could work concurrently, if yes, use the max value of them. Otherwise, use the sum of them.\\
calcuate the AMAT for \texttt{Mr.Bird}.
\section*{Algorithm Design}
Now we focus on a $N=n*n$ mesh(you may google to find what shape it is since it is hard to plot). Every link has only 1 virtual channel.
\subsection*{Part A}
Identify whether the following policies are dead-lock free, and provide a proof.\\
\subsubsection*{1}
We choose a minimal path at random sequence of first $X$ then $Y$ or first $Y$ then $X$.
\subsubsection*{2}
Less on random, diliver a message from $(x_1,y_1)\to(x_2,y_2)$. If $x_2>x_1,y_2>y_1$, then first $X$ then $Y$. If $x_2<x_1,y_2<y_1$, then first $Y$ then $X$. Other wise random deliver.
\subsubsection*{3}
See the PPT. Hard to describe.
\subsection*{Part B}
Now we focus on a 3*3 mesh.
\subsubsection*{1}
Each point randomly deliver message to all points, calculate the average distance.
\subsubsection*{2}
label the point with row major($0\to 8$), calculate the average distance.
\subsubsection*{3}
Is there a better permutation for the labels that has a smaller average distance? If yes, find it. If no, prove it.
\section*{Problem 5 Bird}
A true master should not only good at do the problem, but also good at predict his score. Give a prediction of your score, you will receive a bouns point if your point is in $\pm 5$.\\
\end{document}